% !TEX root = ejemplo.tex

\chapter{Paquetes en la clase}

\section{amsmath y mathtools}
Aunque en el título de la sección se menciona \texttt{amsmath} en la clase
sólo se carga el paquete \texttt{mathtools}. Esto se debe a que
\texttt{mathtools} carga al paquete \texttt{amsmath}, así lo podemos pensar
como una extensión. De manera más precisa \texttt{mathtools} es resultado de
corregir algunos bugs de \texttt{amsmath} y añadir algunas otras funciones.

Un ejemplo específico de estos paquetes es la creación de operadores. En el
modo matemático de \LaTeX{} una letra representa ua variable y dos variables
juntas representa la multiplicación de estas. De esta manera \(xy\,a\)
representa la multiplicación de las variables \(x\), \(y\) y \(a\), mientra
que \(\operatorname{xy}a\) representa al operador \(\operatorname{xy}\)
aplicado a la variable \(a\). En otras palabras, un cambio en el alfabeto
debería implicar un cambio en el significado. Por esta razón deberíamos
escribir operadores de una forma especial y no sólo escribiendo su nombre en
modo matemático.

Por ejemplo, para escribir el supremo del conjunto \(A\) se debe escribir
\verb|\sup A| y su resultado es \(\sup A\), de donde es claro que los
operadores están en letras \textit{upright} (redondas en español). Ya están
definidos los operadores más comunes, pero si se quisiera definir uno nuevo
se escribe en el preámbulo \verb|\DeclareMathOperator{\idem}{Idem}| para
definir, por ejemplo, los idempotentes de un anillo, así \verb|\idem(R)|
genera \(\idem(R)\).

Además \texttt{amsmath} define ambientes matemáticos como
\texttt{align}, \texttt{gather}, \texttt{cases}, \texttt{equation},
\texttt{array}, las variantes de \texttt{matrix}, etc.\ y \texttt{mathtools}
corrige algunos errores en ellos, por ejemplo en \texttt{gather} o añade
algunas opciones, por ejemplo en los ambientes de matriz se puede
especificar la alineación de las columnas de la misma manera que en
\texttt{array}.

Una función de \texttt{mathtools} que no tiene \texttt{amsmath} es la creación de delimitadores que se pueden ajustar al tamaño del contenido, por ejemplo para hacer un delimitador para el valor absoluto se escribe
\verb|\DeclarePairedDelimiter\abs{\lvert}{\rvert}| este
tiene un argumento más que el operador ya que hay que decir qué símbolo
\enquote{abre} y que símbolo \enquote{cierra}. No es necesario que concuerde
un símbolo que abre con su pareja de cierre, como los paréntesis. Por
ejemplo se podría definir un delimitador para intervalos abiertos por la
izquierda y cerrados por la derecha. La deferencia de estos delimitadores se
puede apreciar con su versión con estrella y con el código

\begin{tabular}{llll}
  \verb!|\sum_{i=1}^{n}a_1|! & \(|\sum\limits_{i=1}^{n}a_1|\), &
  \verb|\lvert\sum_{i=1}^{n}a_1\rvert| &
  \(\lvert\sum\limits_{i=1}^{n}a_1\rvert \) \\[4ex]
  \verb|\abs{\sum_{i=1}^{n}a_1}| & \(\abs{\sum\limits_{i=1}^{n}a_1}\), &
  \verb|\abs*{\sum_{i=1}^{n}a_1}| & \(\abs*{\sum\limits_{i=1}^{n}a_1}\)
\end{tabular}

\verb|\DeclarePairedDelimiter| tiene una versión extendida para que pueda
tomar argumentos, como \verb|\newcommand{}[]{}|. Esta versión extendida se
escribe \verb|\DeclarePairedDelimiterX| y con ella podemos hacer, por
ejemplo, conjuntos  con delimitadores y la línea de \enquote{tal que} que
crezcan correctamente (ver su definición en \texttt{ejemplo.tex})
\[
  \Set*{ x \in X \given \frac{\sqrt{x}}{x^2+1} > 1 }
\]


\section{amsthm}
Este paquete provee mejoras útiles para las definiciones de teoremas
(\LaTeX{} puede crear teoremas sin necesidad de paquetes) y
define un ambiente de demostración (esto no lo hace \LaTeX{}). Con este
paquete se pueden usar y definir estilos de teoremas de forma fácil. En la clase se definieron los siguientes ambientes:
\begin{center}
  \autocols{c}{3}{l}{\texttt{definicion}, \texttt{lema}, \texttt{teorema}, \texttt{proposicion}, \texttt{corolario}, \texttt{observacion}}
\end{center}
Con la salida esperada del nombre del ambiente. Por ejemplo:

\begin{definicion}
  Una función \(f\colon X\to Y\) es continua si para cualquier abierto
  \(V\subseteq Y\) se tiene que \(f^{-1}(V)\subseteq X\) es abierto.
\end{definicion}

\begin{teorema}[Fermat]%
\label{teo:fermat}
  La ecuación \(x^n + y^n = z^n\) con \(n\geq 3\) no tiene soluciones no triviales en \(\mathbb{Z}\).
\end{teorema}
\begin{proof}
  He descubierto una demostración maravillosa de esto, que este margen es
  demasiado estrecho para contener.
\end{proof}

Se modificó el estilo del ambiente demostración para que su salida sea
similar a la de los resultados, una demostración es tan importante como el
enunciado. Nuestra redefinición del ambiente de demostración se basa en la
definición de \texttt{amsthm} así tiene lo necesario para comportarse bien
con el símbolo \(\qed \). Es decir, cuando se termina una demostración con
una ecuación u otro tipo de ambiente habrá un espacio vertical no deseado
entre el final del texto y el símbolo de fin de la demostración. Este
espacio se evita con el comando \verb|\qedhere|, pero si el ambiente no
cuenta con la definición correcta seguirá apareciendo este espacio vertical.
Un ejemplo de cómo funciona \verb|\qedhere|, para mostrar la diferencia se
han hecho dos demostraciones iguales

\begin{teorema}
  Un resultado importante.
\end{teorema}
\begin{proof}
  Se sigue de la siguiente ecuación (sin usar \verb|\qedhere|)
  \[
    \sum_{n\geq 1}\frac{1}{n}=-\frac{1}{12}.
  \]
\end{proof}
\begin{proof}[¿Es otra demostración?]
  Se sigue de la siguiente ecuación (usando \verb|\qedhere|)
  \[
    \sum_{n\geq 1}\frac{1}{n}=-\frac{1}{12}. \qedhere
  \]
\end{proof}

Cada \enquote{teorema} nuevo crea un contador. En este ejemplo la cuenta de
teoremas se reiniciará al iniciar un nuevo capítulo, este es el efecto del
comando opcional \texttt{[chapter]} en la definición de \texttt{teorema}.
Además, el contador de las definiciones será el mismo que el de los teoremas
(notar que aparece definición 4.1, teorema 4.2), esto lo hace el parámetro
opcional \texttt{[teorema]} en la definición de \texttt{definicion}. Esta
cuenta de teoremas sirve para hacer referencia a resultados o definiciones
en el futuro. Al teorema le pusimos una etiqueta con el comando
\verb|\label{teo:fermat}| que luego se puede hacer referencia con
\verb|\ref{{teo:fermat}}|. Una buena práctica es separar la referencia de la
palabra anterior con un espacio irrompible \verb|~|, de esta forma cuando
escribimos \enquote{por el teorema~\ref{teo:fermat}} \LaTeX{} no podrá romper un
renglón dejando la palabra \enquote{teorema} en un renglón y el número \enquote{4.2} en
otro.

Nota que el estilo \texttt{plain} pone el cuerpo del teorema en itálicas
mientras que el estilo \texttt{definition} no, como puede verse en el
enunciado de la definición y de los teoremas.

En el preámbulo de \texttt{ejemplo.tex} está un ejemplo de cómo crear un
estilo nuevo de \enquote{teorema}. En este se creó un ambiente para los axiomas
usando un estilo diferente que llevará una cuenta independiente de las
definiciones anteriores, pues no aparece la opción \texttt{[teorema]} en su
definición.

\begin{axioma}
  Para toda \(f\colon D\to R\) existe una y sólo una \(b\in R\) tal que para
  cualquier \(d\in D\)
  \[
    f(d)=f(0)+d\cdot b.
  \]
\end{axioma}

Adicionalmente, se podría hacer las definiciones de estos ambientes de
teorema y demostración como un nuevo ambiente. El ejemplo, compilando con
Lua\LaTeX, con tipo de letra sans en~\ref{textosans} tiene lo necesario para
crear un ambiente y contador con las características como las de los
ambientes de \texttt{amsthm}.

Además, junto con \texttt{amsthm} se podría usar el paquete
\texttt{thmtools} que permite hacer estilos de teoremas, repetirlos, hacer
una lista de teoremas, entre otros. La documentación de \texttt{thmtools} es
fácil de leer ya que consta de ejemplos, así es posible ver la salida de los
comandos que define.

Finalmente, otro paquete común para el manejo de teoremas y demostraciones
es \texttt{ntheorem}. Cada uno tiene sus ventajas y desventajas y no es
fácil elegir uno sobre otro. En este documento se eligió \texttt{amsthm}
porque es (posiblemente) el más común.


\section{babel y polygossia}%
\label{sec:babel}
Para el soporte de idiomas elegimos \texttt{babel} ya que tiene mayor soporte que otros paquetes similares, como \texttt{polygossia}.

\texttt{babel} se carga con las opciones \texttt{spanish} y
\texttt{mexico}. La opción \texttt{mexico} hace una localización más similar
a la que se usa en México, como su nombre lo indica. Entre las cosas que
hace podemos mencionar que cambia el nombre \enquote{cuadro} por \enquote{tabla},
prioriza comillas y usa el punto decimal en lugar de la coma como puede
verse en la compilación con \(\pi=3.141592\ldots \) Además se
cargó la opción \texttt{es-noindentfirst} para evitar el sangrado en el
primer párrafo después de un título, como se hace en textos en español. Por
último se deshabilitaron los \texttt{shorthands} o taquigrafías para evitar
algunos conflictos con la sintaxis de otros paquetes, por ejemplo las
comillas en \texttt{xy-pic} o en \texttt{tikzcd}.

Al usar el idioma español \texttt{babel} se encargará de traducir todo, por
ejemplo la palabra \enquote{capítulo} o \enquote{figura} como se puede ver
en el documento. También traduce y acentúa los operadores, por ejemplo
\(\max A\) o \(\lim f\). Pero en algunos casos se decidió crear un comando
nuevo como en el caso del seno:
\[
  \sin(\alpha)\ne\sen(\alpha)
\]


Otra opción para el manejo de idiomas en Xe\LaTeX{} o Lua\LaTeX{} es
\texttt{polyglossia}. Este paquete se creó cuando \texttt{babel} había
dejado de tener mantenimiento y uno de sus objetivos es simplificar su
trabajo en estos motores. Este paquete también tiene una variante
\texttt{mexico} (no compatible con \texttt{biber}), además de que se puede
elegir un poco más el comportamiento de los operadores con las opciones:
\begin{description}
  \item[\texttt{accented}] para acentuar los operadores como \texttt{babel}.
  \item[\texttt{spaced}] para dar un espacio corto entre el operador y el objeto al que se le aplica, de nuevo, como \texttt{babel}.
  \item[\texttt{all}] para hacer las dos anteriores más la localización de \verb|\sin| \verb|\tan| \verb|\sinh| y \verb|\tanh|.
  \item[\texttt{none}] para no hacer ninguna localización.
\end{description}
Con Lua\LaTeX{} se podría cargar \texttt{polygossia} con la siguiente
configuración
\begin{flushleft}
  \verb|\usepackage{polyglossia}| \\
  \verb|\setdefaultlanguage[spanishoperators=all]{spanish}|
\end{flushleft}

De nuevo, para evitar el sangrado del primer párrafo después de un título
se debería añadir el comando
\begin{flushleft}
  \verb|\PolyglossiaSetup{spanish}{indentfirst=false}|
\end{flushleft}

Tanto \texttt{babel} como \texttt{polyglossia} son buenas opciones para el
manejo de idiomas y ambos tienen ventajas y desventajas en algunos aspectos
sobre el otro. Aún así elegimos usar \texttt{babel} por ser más conocido y tener un mejor soporte.


\section{microtype}
Este paquete habilita los aspectos micro-tipográficos del documento. Algunos
de ellos ya son hechos por \TeX{} como justificación y la separación
silábica (cortes de palabra). En general las opciones que se le pasaron a
este paquete sirven para calcular la expansión de letras, palabras y
permitir que algunos caracteres, como el guión de un corte de palabra,
salgan del margen. Algunas de estas mejoras se pueden hacer con
\texttt{fontspec}, por ejemplo poniendo la opción \texttt{LetterSpacing} al
cargar una fuente, pero hay que calcular las cosas manualmente. En cambio
\texttt{microtype} toma todas estas decisiones por nosotros y lo hace
de acuerdo al idioma que se la haya pasado a \texttt{babel} o
\texttt{polyglossia}. El resultado es un pdf que luce tipográficamente mejor.


\section{siunitx}
Este paquete tiene como objetivo implementar la escritura de cantidades
físicas de acuerdo con las reglas del sistema internacional de medidas.\footnote{\url{https://www.bipm.org/en/measurement-units/}} Aunque hay un desacuerdo con la forma de espaciar las cantidades y las unidades debido a la mala traducción del francés (idioma de la versión oficial del manual del sistema internacional de medidas) al inglés (idioma de donde se basaron para la creación del paquete). En el francés dice que deberían separase con un espacio y en el inglés dice que se separan con un \textit{thin space}. Así, el espaciado no es el correcto pero el error puede ser productivo ya que un menor espaciado crea una relación más estrecha entre cantidad y unidad. Algunos ejemplos de este paquete están en la siguiente tabla:
\begin{center}
  \begin{tabular}{lc}
    \toprule
    Comando & Resultado \\
    \midrule
    \verb|\ang{1;2;3}| & \ang{1;2;3}\\
    \verb|\si{\gram\per\cubic\centi\metre}| & \si{\gram\per\cubic\centi\metre}\\
    \verb|\si{kg.m/s^2}| & \si{kg.m/s^2}\\
    \verb|\SI{.23e7}{\candela}| & \SI{.23e7}{\candela} \\
    \verb|\SI[mode=text]{1.23}{J.mol^{-1}.K^{-1}}| & \SI[mode=text]{1.23}{J.mol^{-1}.K^{-1}} \\
    \verb|\SI[per-mode=symbol]{1.99}[\$]{\per\kilogram}| & \SI[per-mode=symbol]{1.99}[\$]{\per\kilogram} \\
    \verb|\SI[per-mode=fraction]{1,345}{\coulomb\per\mole}| & \SI[per-mode=fraction]{1,345}{\coulomb\per\mole}\\
    \bottomrule
  \end{tabular}
\end{center}


\section{csquotes}%
\label{se:csq}
Este paquete no fue cargado por defecto en la clase pero es recomendado
para la bibliografía, sobretodo si se usa \texttt{biblatex}, como en nuestro
ejemplo.

Este paquete provee algunas facilidades en citas y ajustar las
comillas al idioma que se quiera. En este documento se uso junto con la
opción \texttt{style=mexican} que carga el estilo de comillas del idioma
español con la variante para México. También define comandos para poner
comillas y hacer citas, por ejemplo \verb|\enquote{comillas}| resulta en
\enquote{comillas}. Este comando se puede anidar para poner las comillas
correctas dentro de otras comillas, por ejemplo
\begin{flushleft}
  \verb|\enquote{Lorem ipsum \enquote{dolor} sit amet}|
\end{flushleft}
resulta en \enquote{Lorem ipsum \enquote{dolor} sit amet}. Puede ser útil
escribir comillas mediante un comando, ya que muchas veces no se usan las
comillas correctas de \LaTeX, estas son \verb|``...''|. Además, da
facilidades para poner citas textuales con el comando
\verb|\textquote[⟨cita⟩][⟨puntuación⟩]{⟨texto⟩}|. También tiene un comando
para citar un bloque de texto
\verb|\blockquote[⟨cita⟩][⟨puntuación⟩]{⟨texto⟩}|. Todas estas funciones
tienen una versión para citar en un idioma extranjero, este idioma también
debe ser cargado en \texttt{babel} (o \texttt{polygossia}).


\section{biblatex}%
\label{sec:bib}
En principio no se cargo por defecto ningún paquete para crear la
bibliografía del documento, pero el que nos parece recomendable es
\texttt{biblatex}. Hay otros paquetes para formar la bibliografía de un
texto, por ejemplo \texttt{natbib} o \texttt{cite}.

Mientras que \texttt{natbib} y \texttt{cite} son muy parecidos, sólo cambian
algunas capacidades y \texttt{natbib} puede hacer técnicamente todo lo que
hace \texttt{cite} más algunas cuantas cosas; \texttt{biblatex} es muy
diferente a estos dos paquetes.

Como este es un texto \enquote{local} en el sentido que su código fuente no
debe cumplir los requerimientos de ningún \textit{journal} o editorial,
entonces es posible usar los paquetes que cada quien considere necesario. La
motivación principal para hacer este ejemplo con \texttt{biblatex} está
en la siguente liga \url{https://tex.stackexchange.com/a/25702/140456},
donde se explican las ventajas de biber sobre bibtex.

La construcción de la base de datos de referencias es básicamente la misma
para cualquier paquete de los anteriores. Hay herramientas gráficas útiles
para hacer esto como JabRef o Zotero.

El paquete \texttt{biblatex} puede manejar muchos tipos de entrada. En el
manual se describe el uso de los siguientes
\autocols{c}{5}{l}{%
  article, book, mvbook, inbook, bookinbook, suppbook, booklet, collection, mvcollection, incollection, suppcollection, dataset, manual, misc, online, patent, periodical, suppperiodical, proceedings, mvproceedings, inproceedings, reference, mvreference, inreference, report, set, software, thesis, unpublished, xdata, custom[a-f], conference, masterthesis, pdhthesis, techreport, www, artwork, audio, bibnote, commentary, image, jurisdiction, legislation, legal, letter, movie, music, performance, review, standard, video
  }
Son demasiados para dar una descripción breve, pero en el manual de
\texttt{biblatex} no sólo se describe para qué se usa cada una sino que
también algunos campos recomendados para algunas de ellas.

La lista de campos que se pueden usar en las entradas es demasiado larga para
escribirla aquí. De nuevo, en el manual se dan los campos disponibles junto
con una descripción breve.

También tiene una lista grade de formas de citación, las más comunes son
\verb|\cite{...}|, \verb|\parencite{...}|, \verb|\textcite{...}| y
\verb|\footcite{...}|. El estilo de las entradas bibliográficas y de la
citación se hacen mediante opciones del paquete. Como la lista de opciones
también es amplia haremos referencia a la siguiente liga
\url{http://tug.ctan.org/info/biblatex-cheatsheet/biblatex-cheatsheet.pdf}
para tener una guía rápida de \texttt{biblatex}.

En este ejemplo hicimos un archivo \texttt{refs.bib} como base de datos de la
bibliografía. Para \enquote{cargarlo} hay que usar el comando
\verb|\addbibresource{refs.bib}|. Este comando sólo acepta un archivo, pero
se pueden cargar más escribiendo todo el comando con cada archivo
\texttt{.bib} que se quiera usar. Para imprimir la bibliografía se pone el
comando \verb|\printbibliography| donde se quiera tener la bibliografía. Por
ejemplo se puede imprimir una bibliografía por capitulo, por tipo
(artículo, libro, etc.), por alguna palabra clave, etc.

En principio sólo imprime las entradas que fueron citadas en el texto, si se
quiere imprimir una entrada que no fue citada se usa el comando
\verb|\nocite{label1,label2,...}| o si se quiere imprimir todas las entradas
del archivo \texttt{.bib} se usa \verb|\nocite{*}|.

Ahora algunos ejemplos de bibliografía. Primero todas las entradas donde
Lawvere es autor, en el archivo \texttt{.bib} use el campo \texttt{keywords}
y para imprimirlas se usó el comando
\begin{flushleft}
  \verb|\printbibliography[keyword=Lawvere,heading=subbibliography,title=Lawvere]|
\end{flushleft}
\printbibliography[keyword=Lawvere,heading=subbibliography,title=Lawvere]

Usamos \texttt{heading=subbibliography} para que el título de la bibliografía
aparezca un nivel más bajo que \enquote{el nivel principal} en este caso el nivel
principal es capítulo, así que el título será impreso como una sección sin
número.

Para imprimir una bibliografía con todos los libros que están en
\texttt{refs.bib} hacemos lo siguiente
\begin{flushleft}
  \verb|\printbibliography[type=book,heading=subbibliography,title=Libros]|
\end{flushleft}
\printbibliography[type=book,heading=subbibliography,title=Libros]

Como puse el ejemplo con muchos tipos de bibliografía y al final pondré toda
la bibliografía usaré el estilo bibliográfico \texttt{alphabetic}.
Otros estilos son \texttt{authoryear-icomp}, \texttt{numeric} (este es común
en matemáticas) \texttt{author-title}, etc. Además de estos y las variantes
ya definidas para apa, chicago, mla, entre otras. Es mucho más fácil definir
o modificar un estilo usando \texttt{biblatex} (ya que se hace con comandos
de \LaTeX{} comunes, al estilo \verb|\renewcommand|) que usando
\texttt{natbib} o \texttt{cite} donde se requiere editar/crear un archivo
\texttt{.bst} con un lenguaje, en principio, muy diferente a \LaTeX.

En el ejemplo de bibliografías múltiples con el estilo \texttt{numeric}
sería ideal usar la opción \texttt{resetnumbers} para que cada bibliografía
empiece en [1], pero esto creará inconsistencias en la numeración de la
bibliografía que pondremos al final del documento.

Una nota final es que el proceso de compilación ahora debe ser el siguiente
\begin{flushleft}
\verb|lualatex MiDocumento.tex|\\
\verb|biber MiDocumento|\\
\verb|lualatex MiDocumento.tex|\\
\verb|lualatex MiDocumento.tex|
\end{flushleft}
