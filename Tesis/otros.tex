% !TEX root = tesis.tex

\chapter{Paquetes en la clase}

\section{amsmath y mathtools}
Aunque en el título de la sección se menciona \texttt{amsmath} en el archivo
principal sólo se carga el paquete \texttt{mathtools}. Esto se debe a que
\texttt{mathtools} carga al paquete \texttt{amsmath}, así lo podemos pensar
como una extensión. De manera más precisa \texttt{mathtools} es resultado de
corregir algunos ``bugs'' de \texttt{amsmath} y añadir algunas otras funciones.

Un ejemplo específico de estos paquetes es la creación de operadores. Por
ejemplo, para escribir el supremo del conjunto \(A\) se debe escribir
\verb|\sup A| y su resultado es \(\sup A\), de donde es claro que los
operadores están en letras \textit{upright}. Ya están definidos los
operadores más comunes, pero si se quisiera definir uno nuevo se escribe en
el preámbulo \verb|\DeclareMathOperator{\id}{Idem}| para definir los
idempotentes de un anillo, por ejemplo \verb|\id(R)| genera \(\id(R)\).

Además de esto están los ambientes matemáticos como \texttt{align},
\texttt{gather}, \texttt{cases}, etc.

Una función de \texttt{mathtools} que no tiene \texttt{amsmath} es la creación
de delimitadores que se pueden ajustar al tamaño del contenido, por ejemplo
para hacer un delimitador para el valor absoluto se escribe
\verb|\DeclarePairedDelimiter\abs{\lvert}{\rvert}| este
tiene un argumento más que el operador ya que hay que decir qué símbolo
``abre'' y que símbolo ``cierra''. La deferencia de este comando se puede
apreciar con el código \verb!|\sum_{i=1}^{n}a_1|!,
\verb|\lvert\sum_{i=1}^{n}a_1\rvert|, \verb|\abs{\sum_{i=1}^{n}a_1}| y
\verb|\abs*{\sum_{i=1}^{n}a_1}|, que su salida es, respectivamente,
\[
  |\sum_{i=1}^{n}a_1| \quad \lvert\sum_{i=1}^{n}a_1\rvert \quad
  \abs{\sum_{i=1}^{n}a_1} \quad \abs*{\sum_{i=1}^{n}a_1}
\]

También podemos hacer conjuntos con delimitadores y la línea de ``tal que''
que crezcan correctamente (ver su definición en \texttt{ejemplo.tex})
\[
  \Set*{ x \in X \given \frac{\sqrt{x}}{x^2+1} > 1 }
\]


\section{amsthm}
Este paquete provee mejoras útiles para las definiciones de teoremas
(\LaTeX{} puede crea teoremas sin necesidad de paquetes) y
define un ambiente de demostración (esto no lo hace \LaTeX{}). Con este
paquete se pueden usar y definir estilos de teoremas de forma fácil. En la calse se definieron los siguinetes ambientes:
\begin{center}
  \autocols{c}{3}{l}{\texttt{definicion}, \texttt{lema}, \texttt{teorema}, \texttt{proposicion}, \texttt{corolario}, \texttt{observacion}}
\end{center}
Con la salida esperada del nombre del ambiente.

\begin{definicion}
  Una función \(f\colon X\to Y\) es continua si para cualquier abierto
  \(V\subseteq Y\) se tiene que \(f^{-1}(V)\subseteq X\) es abierto.
\end{definicion}

\begin{teorema}[Fermat]%
\label{teo:fermat}
  La ecuación \(x^n + y^n = z^n\) con \(n\geq 3\) no tiene soluciones no triviales en \(\symbb{Z}\).
\end{teorema}
\begin{proof}
  He descubierto una demostración maravillosa de esto, que este margen es
  demasiado estrecho para contener.
\end{proof}

Se modificó el estilo del ambiente demostración para que su salida sea similara a la de los resultados —Una demostración es tan importante como el
enunciado—. Nuestra redefinición del ambiente de demostración tiene lo necesario para comportarse bien con el símbolo \(\qed \). Es decir, cuando se termina una demostración con una ecuación u otro tipo de ambiente habrá
un espacio vertical indeseado entre el final del texto y el símbolo de fin de la demostración. Este espacio se evita con el comando \verb|\qedhere|, pero si el ambiente no cuenta con la definición correcta seguirá
apareciendo este espacio vertical. Un ejemplo de cómo funciona \verb|\qedhere|, para mostrar la diferencia se han hecho dos demostraciones iguales

\begin{teorema}
  Un resultado importante.
\end{teorema}
\begin{proof}
  Se sigue de la siguiente ecuación (sin usar \verb|\qedhere|)
  \[
    \sum_{n\geq 1}\frac{1}{n}=-\frac{1}{12}.
  \]
\end{proof}
\begin{proof}[¿Es realmente otra demostración?]
  Se sigue de la siguiente ecuación (usando \verb|\qedhere|)
  \[
    \sum_{n\geq 1}\frac{1}{n}=-\frac{1}{12}. \qedhere
  \]
\end{proof}

Cada ``teorema'' nuevo crea un contador. En este ejemplo la cuenta de
teoremas se reiniciará al iniciar un nuevo capítulo, este es el efecto del
comando opcional \texttt{[chapter]} en la definición de \texttt{teorema}.
Además, el contador de las definiciones será el mismo que el de los teoremas
(notar que aparece definición 4.1, teorema 4.2), esto lo hace el parámetro
opcional \texttt{[teorema]} en la definición de \texttt{definicion}. Esta
cuenta de teoremas sirve para hacer referencia a resultados o definiciones
en el futuro. Al teorema le pusimos una etiqueta con el comando
\verb|\label{teo:fermat}| que luego se puede hacer referencia con
\verb|\ref{{teo:fermat}}|. Una buena práctica es separar la referencia de la
palabra anterior con un espacio irrompible \verb|~|, de esta forma cuando
escribimos ``por el teorema~\ref{teo:fermat}'' \LaTeX{} no podrá romper un
renglón dejando la palabra ``teorema'' en un renglón y el número ``4.2'' en
otro.

Nota que el estilo \texttt{plain} pone el cuerpo del teorema en itálicas
mientras que el estilo \texttt{definition} no, como puede verse en el
enunciado de la definición y de los teoremas.

En el preámbulo de \texttt{ejemplo.tex} está un ejemplo de cómo crear un
estilo nuevo de ``teorema''. En este se creó un ambiente para los axiomas
usando un estilo diferente que llevará una cuenta independiente de las
definiciones anteriores, pues no aparece la opción \texttt{[teorema]} en su
definición.

\begin{axioma}
  Para toda \(f\colon D\to R\) existe una y sólo una \(b\in R\) tal que para
  cualquier \(d\in D\)
  \[
    f(d)=f(0)+d\cdot b.
  \]
\end{axioma}

Adicionalmente, se podría hacer las definiciones de estos ambientes de
teorema y demostración como un nuevo ambiente. El ejemplo con tipo de letra
sans en~\ref{textosans} tiene lo necesario para crear un ambiente y contador
con las características como las de los ambientes de \texttt{amsthm}.

Finalmente, otro paquete común para el manejo de teoremas y demostraciones
es \texttt{ntheorem}. Cada uno tiene sus ventajas y desventajas y no es
fácil elegir uno sobre otro. En este documento se eligió \texttt{amsthm}
porque es (posiblemente) el más común.


\section{babel}%
\label{sec:babel}
Para el soporte de idiomas elegimos \texttt{babel} con las opciones
\texttt{spanish} y \texttt{mexico}. La opción \texttt{mexico} hace una
localización más similar a la que se usa en México, como su nombre lo indica.
Entre las cosas que hace podemos mencionar que cambia el nombre ``cuadro''
por ``tabla'', prioriza comillas y usa el punto decimal en lugar de la coma
como puede verse \(\pi=3.141592\ldots \)

Al usar el idioma español \texttt{babel} se encargará de traducir todo, por ejemplo la palabra ``capítulo'' o ``figura'' como se puede ver en el documento. También traduce y acentúa los operadores, por ejemplo \(\max A\) o \(\lim f\), pero en algunos casos se decidió crear un comando nuevo como en el caso del seno:
\[
  \sin(\alpha)\ne\sen(\alpha)
\]

Otra opción para el manejo de idiomas en Xe\LaTeX{} o Lua\LaTeX{} es
\texttt{polyglossia}. Este paquete se creó cuando \texttt{babel} había
dejado de tener mantenimiento con el objetivo de simplificar su trabajo en
estos motores. Si se elige usar \texttt{polyglossia} este paquete también
tiene una variante \texttt{mexico}, además de que se puede elegir un poco
más el comportamiento de los operadores con las opciones:
\begin{description}
  \item[\texttt{accented}] para acentuar los operadores como \texttt{babel}.
  \item[\texttt{spaced}] para dar un espacio corto entre el operador y el objeto al que se le aplica, de nuevo, como \texttt{babel}.
  \item[\texttt{all}] para hacer las dos anteriores más la localización de \verb|\sin| \verb|\tan| \verb|\sinh| y \verb|\tanh|.
  \item[\texttt{none}] para no hacer ninguna localización.
\end{description}
De esta forma podríamos usar \texttt{polyglossia} con la siguiente
configuración
\begin{flushleft}
  \verb|\usepackage{polyglossia}| \\
  \verb|\setdefaultlanguage[spanishoperators=all]{spanish}|
\end{flushleft}

Tanto \texttt{babel} como \texttt{polyglossia} son buenas opciones para el
manejo de idiomas y es difícil elegir uno sobre el otro. En este caso elegí
\texttt{babel} por ser más conocido (lleva mucho más tiempo existiendo) y
por ser el más estable (\texttt{polyglossia} está en desarrollo, aunque lo
he usado en la mayoría de mis documentos y no he tenido problemas con la
configuración como la de arriba).


\section{microtype}
Este paquete habilita los aspectos micro-tipográficos del documento. Algunos
de ellos ya son hechos por \TeX{} como justificación y la separación
silábica (cortes de palabra). En general las opciones que se le pasaron a
este paquete sirven para calcular la expansión de letras, palabras y
permitir que algunos caracteres, como el guión de un corte de palabra,
salgan del margen. Algunas de estas mejoras se pueden hacer con
\texttt{fontspec}, por ejemplo poniendo la opción \texttt{LetterSpacing} al
cargar una fuente, pero hay que calcular las cosas manualmente. En cambio
\texttt{microtype} toma todas estas decisiones por nosotros y lo hace
de acuerdo al idioma que se la haya pasado a \texttt{babel}. El resultado es
un pdf que luce tipográficamente mejor.


\section{siunitx}
Este paquete tiene como objetivo implementar la escritura de cantidades
físicas de acuerdo con las reglas del sistema internacional de medidas\footnote{\url{https://www.bipm.org/en/measurement-units/}}. Aunque hay un desacuerdo con la forma de espaciar las cantidades y las unidades debido a la mala traducción del francés (idioma de la versión oficial del manual del sistema internacional de medidas) al inglés (idioma de donde se basaron para la creación del paquete). En el francés dice que deberían separase con un espacio y en el inglés dice que se separan con un \textit{thin space}. Así, el espaciado no es el correcto pero el error puede ser productivo ya que un menor espaciado crea una relación más estrecha entre cantidad y unidad. Algunos ejemplos de este paquete están en la siguiente tabla:
\begin{center}
  \begin{tabular}{lc}
    \toprule
    Comando & Resultado \\
    \midrule
    \verb|\ang{1;2;3}| & \ang{1;2;3}\\
    \verb|\si{\gram\per\cubic\centi\metre}| & \si{\gram\per\cubic\centi\metre}\\
    \verb|\si{kg.m/s^2}| & \si{kg.m/s^2}\\
    \verb|\SI{.23e7}{\candela}| & \SI{.23e7}{\candela} \\
    \verb|\SI[mode=text]{1.23}{J.mol^{-1}.K^{-1}}| & \SI[mode=text]{1.23}{J.mol^{-1}.K^{-1}} \\
    \verb|\SI[per-mode=symbol]{1.99}[\$]{\per\kilogram}| & \SI[per-mode=symbol]{1.99}[\$]{\per\kilogram} \\
    \verb|\SI[per-mode=fraction]{1,345}{\coulomb\per\mole}| & \SI[per-mode=fraction]{1,345}{\coulomb\per\mole}\\
    \bottomrule
  \end{tabular}
\end{center}


\section{biblatex}%
\label{sec:bib}
En principio no se eligió ningún paquete para crear la bibliografía del documento, pero el que nos parece recomendable es \texttt{biblatex}. Hay otros paquetes para formar la bibliografía de un texto, por ejemplo
\texttt{natbib} o \texttt{cite}.

Mientras que \texttt{natbib} y \texttt{cite} son muy parecidos, sólo cambian
algunas capacidades y \texttt{natbib} puede hacer técnicamente todo lo que
hace \texttt{cite} más algunas cuantas cosas; \texttt{biblatex} es muy
diferente a estos dos paquetes.

Como este es un texto ``local'' en el sentido que su código fuente no debe
cumplir los requerimientos de ningún \textit{journal} o editorial, entonces
es posible usar los paquetes que cada quien considere necesario. La
motivación principal para hacer este ejemplo con \texttt{biblatex} está
en el siguiente link \url{https://tex.stackexchange.com/a/25702/140456},
donde se explican las ventajas de biber sobre bibtex.

La construcción de la base de datos de referencias es básicamente la misma
para cualquier paquete de los anteriores. Hay herramientas gráficas útiles
para hacer esto como JabRef o Zotero.

\texttt{biblatex} puede manejar muchos tipos de entrada. En el manual se
describe el uso de los siguientes
\autocols{c}{5}{l}{%
  article, book, mvbook, inbook, bookinbook, suppbook, booklet, collection, mvcollection, incollection, suppcollection, dataset, manual, misc, online, patent, periodical, suppperiodical, proceedings, mvproceedings, inproceedings, reference, mvreference, inreference, report, set, software, thesis, unpublished, xdata, custom[a-f], conference, masterthesis, pdhthesis, techreport, www, artwork, audio, bibnote, commentary, image, jurisdiction, legislation, legal, letter, movie, music, performance, review, standard, video
  }
Son demasiados para dar una descripción breve, pero en el manual de
\texttt{biblatex} no sólo se describe para qué se usa cada una sino que
también algunos campos recomendados para algunas de ellas.

La lista de campos que se pueden usar en las entradas es demasiado larga para
escribirla aquí. De nuevo, en el manual se dan los campos disponibles junto
con una descripción breve.

También tiene una lista grade de formas de citación, las más comunes son
\verb|\cite{...}|, \verb|\parencite{...}|, \verb|\textcite{...}| y
\verb|\footcite{...}|. El estilo de las entradas bibliográficas y de la
citación se hacen mediante opciones del paquete. Como la lista de opciones
también es amplia haré referencia a la siguiente liga
\url{http://tug.ctan.org/info/biblatex-cheatsheet/biblatex-cheatsheet.pdf}
para tener una guía rápida de \texttt{biblatex}.

En este ejemplo hice un archivo \texttt{refs.bib} como base de datos de la
bibliografía. Para ``cargarlo'' hay que usar el comando
\verb|\addbibresource{refs.bib}|. Este comando sólo acepta un archivo, pero
se pueden cargar más escribiendo todo el comando con cada archivo
\texttt{.bib} que se quiera usar. Para imprimir la bibliografía se pone el
comando \verb|\printbibliography| donde se quiera tener la bibliografía. Por
ejemplo se puede imprimir una bibliografía por capitulo, por tipo
(artículo, libro, etc.), por alguna palabra clave, etc.

En principio sólo imprime las entradas que fueron citadas en el texto, si se
quiere imprimir una entrada que no fue citada se usa el comando
\verb|\nocite{label1,label2,...}| o si se quiere imprimir todas las entradas
del archivo \texttt{.bib} se usa \verb|\nocite{*}|.

Por último es recomendable usar el paquete \texttt{csquotes} junto con
\texttt{biblatex} para tener algunas facilidades en citas y ajustar las
comillas al idioma que se quiera. En este documento se uso junto con la
opción \texttt{autostyle} que carga el estilo de comillas del idioma que se
haya pasado a babel, es este caso español. Como puede verse en la
bibliografía en México no usamos las comillas de México, más bien usamos las
comillas inglesas. Para hacer el cambio a comillas inglesas se debe
sustituir \texttt{autostyle} por \texttt{style=american}

Ahora algunos ejemplos de bibliografía. Primero todas las entradas donde
Lawvere es autor, en el archivo \texttt{.bib} use el campo \texttt{keywords}
para lograr esto se usó el comando
\begin{flushleft}
  \verb|\printbibliography[keyword=Lawvere,heading=subbibliography,title=Lawvere]|
\end{flushleft}
\printbibliography[keyword=Lawvere,heading=subbibliography,title=Lawvere]

Usamos \texttt{heading=subbibliography} para que el título de la bibliografía
aparezca un nivel más bajo que ``el nivel principal'' en este caso el nivel
principal es capítulo, así que el título será impreso como una sección sin
número.

Para imprimir una bibliografía con todos los libros que están en
\texttt{refs.bib} hacemos lo siguiente
\begin{flushleft}
  \verb|\printbibliography[type=book,heading=subbibliography,title=Libros]|
\end{flushleft}
\printbibliography[type=book,heading=subbibliography,title=Libros]

Como puse el ejemplo con muchos tipos de bibliografía y al final pondré toda
la bibliografía usaré el estilo bibliográfico \texttt{alphabetic}.
Otros estilos son \texttt{authoryear-icomp}, \texttt{numeric} (este es común
en matemáticas) \texttt{author-title}, etc. Además de estos y las variantes
ya definidas para apa, chicago, mla, etc. Es mucho más fácil definir o
modificar un estilo usando \texttt{biblatex} (ya que se hace con comandos de
\LaTeX{} comunes, al estilo \verb|\renewcommand|) que usando \texttt{natbib}
o \texttt{cite} donde se requiere editar/crear un archivo \texttt{.bst} con
un lenguaje, en principio, muy diferente a \LaTeX.

En el ejemplo de bibliografías múltiples con el estilo \texttt{numeric}
sería ideal usar la opción \texttt{resetnumbers} para que cada bibliografía
empiece en [1], pero esto creará inconsistencias en la numeración de la
bibliografía que pondré al final del documento.

Una nota final es que el proceso de compilación ahora debe ser el siguiente
\begin{flushleft}
\verb|lualatex MiDocumento.tex|\\
\verb|biber MiDocumento|\\
\verb|lualatex MiDocumento.tex|\\
\verb|lualatex MiDocumento.tex|
\end{flushleft}
